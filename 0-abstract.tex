\abstractde{
\par Das regelmäßige Überprüfen von High Performance Computing (HPC) Cluster bezüglich Performance-Änderungen von einzelnen Recheneinheiten oder des gesamten Systems ist notwendig um Probleme frühzeitig zu erkennen und zu beheben. Solche  können durch fehlerhafte Konfiguration, falsch installierte Software oder defekte Hardware während des Betriebs durch Updates oder Ermüdung auftreten.
\par Die geminderte Leistungsfähigkeit einzelner Knoten könnte jedoch im Fall von hoch-parallelen Anwendungen mit einigen sequentiellen Abschnitten zu einer signifikanten Laufzeiterhöhung führen, da der auf den schwachen Knoten ausgeführte Anwendungsteil die Ausführung auf den restlichen Knoten verzögert. Ohne eine aufwendige Untersuchung der Anwendungsperformance könnte es sein, dass eine solche nachteilige Ausführung unerkannt bleibt.
\par Die Lösung für dieses Problem könnte ein System sein, dass das HPC Cluster regelmäßigen Tests unterzieht und deren Ergebnisse analysiert ohne die Verfügbarkeit des Clusters erheblich einzuschränken. Entsprechende Tests können Anwendungen sein, die bestimmte Hardwarekomponenten überprüfen oder die Leistungsfähigkeit des Clusters in speziellen Anwendungsszenarien dokumentieren. Eine andere Form der Performance-Analyse ist das permanente Auslesen und Sammeln von Systeminformationen, was als Monitoring bezeichnet wird.
\par Mithilfe eines solchen Systems könnten Systemadministratoren schnell erkennen, ob einzelne Knoten im Vergleich zu Knoten mit identischer Hardware, weniger Rechenleistung zur Verfügung stellen. Auch könnten die Veränderungen durch ein Software-Update Einfluss auf die Performance des Clusters in speziellen Anwendungsszenarien haben und so festgestellt werden. Dies würde es ermöglichen frühzeitig defekte Knoten zu isolieren oder Empfehlungen bezüglich der Verwendung von Softwarepaketen auszusprechen.
\par In meiner Arbeit werde ich bisherige Implementierungen und Praktiken aus den Bereichen der \\Anwendungsperformance-Analyse, dem System-Monitoring und der Performance-Analyse einzelner Knoten vorstellen und diese anhand der Eigenschaften ihrer Funktionalität, Anwendbarkeit, Nutzerfreundlichkeit und Erweiterbarkeit miteinander vergleichen. }
