
\par Beschreibe existierende Tools und ordne diese den Kategorien Monitoring, Experiment-basiert und Application Perofrmance zu. In diesen Bereichen zeige Eigenheiten der Tools bzgl. Möglichkeiten der Performance Messung und Analyse auf. Verdeutliche wie, wo und von wie vielen Nutzern diese Tools aktuell verwendet werden. 



\section{Monitoring Tools}

\subsection{Dataheap}
2012


\subsection{Ganglia}
Von der Universität Berkeley von Kalifornien zusammen mit Intel im Jahr 2004 eingeführtes Tool zur parallellen Datenerfassung auf verteilten Systemen.


\par Durch den einfachen Aufbau und der daraus resultierende schnellen Installation, ist Ganglia zu einem beliebten Monitoring-Werkzeug geworden. Mithilfe seiner erweiterbaren Metriken, kann es an jede beliebige Anwendung angepasst werden und ist dafür ausgelegt auf Systemen mit bis zu 2000 Knoten zu skalieren. 
\par Trotz des relativen Alters der Software, wird das Tool noch immer auf tausenden HPC Systemen verwendet \cite{GangliaWebsite}. Ein limitierender Faktor von Ganglia ist allerdings seine maximale Update-Frequenz von höchstens einer Sekunde, was in Szenarien die eine größere zeitliche Auflösung benötigen, nicht mehr ausreicht.



\subsection{PerfMiner}

\subsection{SuperMon}

\subsection{Lightweight Distributed Metric Service}

\subsection{ClusterProbe}

\subsection{Disparity}

\subsection{NWPerf}



\section{Experiment-basierte Performance Analyse Tools}

\section{Application Performance Analyse Tools}

\subsection{Tracing}

\subsection{Profiling}

\subsection{Visualization}
