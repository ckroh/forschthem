\documentclass[german,beleg,zihtitle,hyperref,utf8]{zihpub}
\usepackage{setspace}
\usepackage{listings}
\usepackage{pgfplots}
%\usepackage{amsfonts}
%\usepackage{amsmath}
%\usepackage{amssymb}
%\usepackage[
%	pdftitle={Characterizing C-state transitions on x86 processors},
%	pdfsubject={},
%	pdfauthor={Christian Kroh},
%	pdfkeywords={}
%]{hyperref} 
%\usepackage{pgfplots}
%\usepackage{isodate}
%\pgfplotsset{compat=1.13}


%\hypersetup{
%	pdfborder={0 0 0},
%	hidelinks
%}

\usepackage{color}
\usepackage{algorithm}
\usepackage[noend]{algpseudocode}
% declaration of the new block



\lstset{ %
  backgroundcolor=\color{white},   % choose the background color; you must add \usepackage{color} or \usepackage{xcolor}; should come as last argument
  basicstyle=\scriptsize,        % the size of the fonts that are used for the code
  breakatwhitespace=false,         % sets if automatic breaks should only happen at whitespace
  breaklines=true,                 % sets automatic line breaking
  captionpos=b,                    % sets the caption-position to bottom
  commentstyle=\color{mygreen},    % comment style
  deletekeywords={...},            % if you want to delete keywords from the given language
  escapeinside={\%*}{*)},          % if you want to add LaTeX within your code
  extendedchars=true,              % lets you use non-ASCII characters; for 8-bits encodings only, does not work with UTF-8
  frame=single,	                   % adds a frame around the code
  keepspaces=true,                 % keeps spaces in text, useful for keeping indentation of code (possibly needs columns=flexible)
  keywordstyle=\color{blue},       % keyword style
%  language=Octave,                 % the language of the code
  morekeywords={*,...},           % if you want to add more keywords to the set
  numbers=left,                    % where to put the line-numbers; possible values are (none, left, right)
  numbersep=5pt,                   % how far the line-numbers are from the code
  numberstyle=\tiny\color{mygray}, % the style that is used for the line-numbers
  rulecolor=\color{black},         % if not set, the frame-color may be changed on line-breaks within not-black text (e.g. comments (green here))
  showspaces=false,                % show spaces everywhere adding particular underscores; it overrides 'showstringspaces'
  showstringspaces=false,          % underline spaces within strings only
  showtabs=false,                  % show tabs within strings adding particular underscores
  stepnumber=1,                    % the step between two line-numbers. If it's 1, each line will be numbered
  stringstyle=\color{mymauve},     % string literal style
  tabsize=2,	                   % sets default tabsize to 2 spaces
  title=\lstname                   % show the filename of files included with \lstinputlisting; also try caption instead of title
}
\usepackage[nomessages]{fp}
\AtBeginDocument{
  \hypersetup{
   pdftitle={Vergleich existierender Tools zur Messung und Analyse der Performance von HPC Clustern},
	pdfsubject={},
	pdfauthor={Christian Kroh},
	pdfkeywords={}
  }
}

\usepackage{subcaption}
\newcommand{\todo}[1]{{\textcolor{red}{ TODO: #1\\}}}

\newcommand{\note}[1]{{\textcolor{mygray}{ \\Note:\\ #1\\}}}
\usetikzlibrary{chains}


\bibfiles{literatur}

%\title{Vergleich existierender Tools zur Performancemessung und -analyse von HPC Clustern}
\title{Vergleich existierender Tools zur Messung und Analyse der Performance von HPC Clustern}
%\faculty{Computer Science}
\betreuer{Dr. Holger Brunst}
\gutachter{Prof. Dr. Wolfgang E. Nagel}
\author{Christian Kroh}
\matno{3755154}
\date{31.03.2018}
%\acknowledgments{sss}


%\copyrighterklaerung{aaa}


\abstractde{
\par Das regelmäßige Überprüfen von High Performance Computing (HPC) Cluster bezüglich Performance-Änderungen von einzelnen Recheneinheiten oder des gesamten Systems ist notwendig um Probleme frühzeitig zu erkennen und zu beheben. Solche  können durch fehlerhafte Konfiguration, falsch installierte Software oder defekte Hardware während des Betriebs durch Updates oder Ermüdung auftreten.
\par Die geminderte Leistungsfähigkeit einzelner Knoten könnte jedoch im Fall von hoch-parallelen Anwendungen mit einigen sequentiellen Abschnitten zu einer signifikanten Laufzeiterhöhung führen, da der auf den schwachen Knoten ausgeführte Anwendungsteil die Ausführung auf den restlichen Knoten verzögert. Ohne eine aufwendige Untersuchung der Anwendungsperformance könnte es sein, dass eine solche nachteilige Ausführung unerkannt bleibt.
\par Die Lösung für dieses Problem könnte ein System sein, dass das HPC Cluster regelmäßigen Tests unterzieht und deren Ergebnisse analysiert ohne die Verfügbarkeit des Clusters erheblich einzuschränken. Entsprechende Tests können Anwendungen sein, die bestimmte Hardwarekomponenten überprüfen oder die Leistungsfähigkeit des Clusters in speziellen Anwendungsszenarien dokumentieren. Eine andere Form der Performance-Analyse ist das permanente Auslesen und Sammeln von Systeminformationen, was als Monitoring bezeichnet wird.
\par Mithilfe eines solchen Systems könnten Systemadministratoren schnell erkennen, ob einzelne Knoten im Vergleich zu Knoten mit identischer Hardware, weniger Rechenleistung zur Verfügung stellen. Auch könnten die Veränderungen durch ein Software-Update Einfluss auf die Performance des Clusters in speziellen Anwendungsszenarien haben und so festgestellt werden. Dies würde es ermöglichen frühzeitig defekte Knoten zu isolieren oder Empfehlungen bezüglich der Verwendung von Softwarepaketen auszusprechen.
\par In meiner Arbeit werde ich bisherige Implementierungen und Praktiken aus den Bereichen der \\Anwendungsperformance-Analyse, dem System-Monitoring und der Performance-Analyse einzelner Knoten vorstellen und diese anhand der Eigenschaften ihrer Funktionalität, Anwendbarkeit, Nutzerfreundlichkeit und Erweiterbarkeit miteinander vergleichen. }

\begin{document}

\chapter{Einleitung}
\par Der Betrieb von High-Performance Computing Systemen ist durch den hohen Energiebedarf 

\par Wieso Cluster Performance ermitteln? -> Vergleich ... Benchmarking, Top500


\par 

\par Beschreibe HPC Cluster Ausfälle ...\cite{Schroeder2010}

\par Auswirkung von schwachen Knoten auf over-all performance von parallelen anwendungen

\par Vergleiche Tools \cite{Kufrin2005,Mucci2005,Kerbyson,Sottile2002,Hoffman2005,Prodan2002,Agelastos2014,Liang1999,Desai2008,Kluge2012,Mooney2004,Massie2004,Huck2005,Worringen2005,Quintero2014,Burtscher2010}

\par Kategorien Monitoring, Experiment Analyse, Application Performance in HPC Performance Analyse

\par Beispiele
\begin{description}
\item[Monitoring] Ganglia, Dataheap, PerfMiner, SuperMon, Lightweight Distributed Metric Service, ClusterProbe, Disparity, NWPerf

\item[Experiment Analyse] PHM (Performance Health Monitor), ZENTURIO, PerfExplorer, PerfBase

\item[Application Performance] Score-P, PerfExpert, HPCToolkit, Open|SpeedShop, PerfSuite, Perf (ZIH-Projekt), Valgrind-based Tools, TAU

\end{description}



\chapter{Tools}

\par Beschreibe existierende Tools und ordne diese den Kategorien Monitoring, Experiment-basiert und Application Perofrmance zu. In diesen Bereichen zeige Eigenheiten der Tools bzgl. Möglichkeiten der Performance Messung und Analyse auf. Verdeutliche wie, wo und von wie vielen Nutzern diese Tools aktuell verwendet werden. 



\section{Monitoring Tools}

\subsection{Dataheap}
2012


\subsection{Ganglia}
Von der Universität Berkeley von Kalifornien zusammen mit Intel im Jahr 2004 eingeführtes Tool zur parallellen Datenerfassung auf verteilten Systemen.


\par Durch den einfachen Aufbau und der daraus resultierende schnellen Installation, ist Ganglia zu einem beliebten Monitoring-Werkzeug geworden. Mithilfe seiner erweiterbaren Metriken, kann es an jede beliebige Anwendung angepasst werden und ist dafür ausgelegt auf Systemen mit bis zu 2000 Knoten zu skalieren. 
\par Trotz des relativen Alters der Software, wird das Tool noch immer auf tausenden HPC Systemen verwendet \cite{GangliaWebsite}. Ein limitierender Faktor von Ganglia ist allerdings seine maximale Update-Frequenz von höchstens einer Sekunde, was in Szenarien die eine größere zeitliche Auflösung benötigen, nicht mehr ausreicht.



\subsection{PerfMiner}

\subsection{SuperMon}

\subsection{Lightweight Distributed Metric Service}

\subsection{ClusterProbe}

\subsection{Disparity}

\subsection{NWPerf}



\section{Experiment-basierte Performance Analyse Tools}

\section{Application Performance Analyse Tools}

\subsection{Tracing}

\subsection{Profiling}

\subsection{Visualization}


\chapter{Anwendungstest}
Teste eine Auswahl von Tools selber aus und beschreibe Erfahrungen.
\section{Monitoring: Dataheap}


\section{Experiment: ?}

\section{Application Performance: Score-P}


\chapter{Aktuelle Entwicklung}
Was sind die aktuell "besten" Tools in dem jeweiligen Bereich bzgl. ausgewählter Eigenschaften.

\chapter{Fazit}



\chapter*{Appendix}
\addtocounter{chapter}{1}
%\section*{Score-P Screenshots}

\begin{figure}[h!]
\centering
 \includegraphics[width=\textwidth]{images/scorep-overview.png}
 \caption{Vampir analysis of Score-P trace}
 \label{fig:fig_over}
\end{figure}

\begin{figure}[h!]
\centering
 \includegraphics[width=\textwidth]{images/scorep-sync.png}
 \caption{Synchronization of Score-P traces via workHard function inVampir}
 \label{fig:fig_sync}
\end{figure}

\begin{figure}[h!]
\centering
 \includegraphics[width=\textwidth]{images/scorep-wakeup.png}
 \caption{Measuring Wake-up latencies in Vampir}
 \label{fig:fig_wake}
\end{figure}


\newpage
\section*{Code Listings}

\subsection*{Linux Kernel Patch}

%\lstinputlisting[language=C, firstnumber=456, firstline=456, lastline=523, label=lst:intel_idle_hsw, caption={Haswell C-state definitions - 4.8.4/drivers/idle/intel\_idle.c}]{listings/intel_idle.c}

\lstinputlisting[language=C, firstnumber=875, firstline=875, lastline=919, label=lst:intel_idle, caption={Recording C-state transitions - 4.8.4/drivers/idle/intel\_idle.c}]{listings/intel_idle.c}

\lstinputlisting[language=C, firstnumber=33, firstline=33, lastline=50, label=lst:include_cpuidle, caption={Data Structure Definition - 4.8.4/include/linux/cpuidle.h}]{listings/include-cpuidle.h}

\lstinputlisting[language=C, firstnumber=446, firstline=446, lastline=696, label=lst:sysfs, caption={Sysfs - 4.8.4/drivers/cpuidle/sysfs.c}]{listings/sysfs.c}

\subsection*{Score-P cpuidle Plugin}

\lstinputlisting[language=C++, label=lst:scorep_cpuidle, caption={Score-P cpuidle Plugin - scorep\_plugin\_cpuidle/scorep\_cpuidle\_plugin.cpp}]{listings/scorep_cpuidle_plugin.cpp}

\subsection*{C-state Trigger Program and Environment}



\lstinputlisting[language=sh, label=lst:cpus_off, caption={Control Script to Disable other CPU cores - control/disable-cpus.sh}]{listings/disable-cpus.sh}
\lstinputlisting[language=sh, label=lst:limit_cstate, caption={Control Script to Deactivate specific Idle States - control/limit\_cstate.sh}]{listings/limit_cstate.sh}




%\lstinputlisting[language=sh, label=lst:ht_setstate, caption={Control Script to Disable Hyper-threading - diana:/usr/bin/ht\_setstate.sh}]{listings/ht_setstate.sh}



%\lstinputlisting[language=sh, label=lst:cpu_setspeed, caption={Control Script to Disable other CPU cores - diana:/usr/bin/cpu\_setspeed.sh}]{listings/cpu_setspeed.sh}


\lstinputlisting[language=sh, label=lst:envars, caption={Environment Variable Definition - control/envvars.sh}]{listings/envars.sh}

\lstinputlisting[language=C, label=lst:trigger, caption={C-state Trigger Program - control/triggeridlestates.c}]{listings/triggeridlestates.c}



\end{document}
